Epilepsy is one of the most common neurological diseases affecting over 50 million people worldwide. Approximately one-third of these patients are refractory to anti-epileptic drugs and surgical resection of epileptic focus remains their only hope for cure. Despite many diagnostic tools, the clear identification of a resectable epileptic focus is still a major bottleneck. This work presents a set of comprehensive quantitative analysis techniques for analyzing and synthesizing infraslow intracranial electroencephalography (iEEG) signals and resting state functional magnetic resonance imaging (rsfMRI) to quantify infra-slow (0.01- 0.1 Hz) network activities, localize seizure onset zones and determine pathological propagation pathways.

Firstly, we examine the existence of a stable network from infra-slow to very high frequencies throughout multiple phases of focal epilepsy using quantitative methods based on spectral Granger causality and graph measures. We show that the strongest infra-slow iEEG (IsEEG) signal correlates highly with the location of the visible seizure focus, and also with that of the strongest high frequency EEG signal, in both the preictal and interictal phases of the epilepsy cycle. 

Secondly, we present a novel quantitative analysis technique to localize the seizure focus by seeding the focus locations from iEEG to rsfMRI. We show that the iEEG electrode contacts with the strongest infraslow iEEG signal correlates with the slow spontaneous blood-oxygen-level-dependent (BOLD) fluctuations in corresponding locations; and those voxels form a highly significant grouping when compared to others throughout the entire brain. This presents an exciting direction in refractory epilepsy to link an invasively recorded iEEG infra-slow network, from a few hypothesized cortical areas, to a non-invasive, whole brain fMRI network.



%These exciting results make a crucial contribution to our understanding of the epilepsy network in unprecedented fine detail and open up new possibilities for localizing seizure onset zones non-invasively.