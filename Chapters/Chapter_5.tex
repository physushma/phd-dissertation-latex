%!TEX root = ../main.tex

\chapter{Summary and Future Directions}
\label{chapter-summary}

%\section{Summary}
In this dissertation, we have provided the first comprehensive quantitative analysis of infraslow iEEG recording from human subjects using intracranial electrodes. We have performed for the first time a quantitative analysis of the information contents of different frequency components in the infraslow iEEG signal. We showed that the strongest infraslow iEEG signal correlates highly with the location of the visible seizure focus, and also with that of the strongest high frequency EEG signal as well, in both the preictal and interictal phases of the epilepsy cycle. In different patients and phases this signal may be more prominent in either total or negative causal flow.

We then demonstrated that the iEEG electrode contacts showing the strongest infraslow iEEG signal correlated with voxels with the strongest resting state fMRI (rsfMRI) signal at the corresponding locations, using similar methods of frequency analysis; and that those those voxels form a highly significant grouping when compared to others throughout the entire brain. Finally, having seeded the epilepsy focus into the rffMRI, we have begun mapping the connections of the epilepsy network throughout the entire brain in unprecedented fine detail.
These exciting results make a crucial contribution to our understanding of epileptic seizure propagation and open up new possibilities for localizing seizure onset zones non-invasively.

%\section{Future Directions}

% \begin{itemize}

% \item Expand the study on more retrospective and prospective patients to standardize our methodology. 

% \item Extensive testing and analysis to determine the optimal methods for characterizing and displaying the results

% \item Explore other quantitative techniques besides Granger causality and graph measures

% \end{itemize}

%Notes to edit: Optimal fine detail mapping of the epilepsy network using the techniques we have pioneered requires further extensive testing and analysis to determine the optimal methods for characterizing and displaying the results. As with other applications of RSfMRI, causal analysis, and graph theory, a great variety of techniques are feasible. Obvious questions of importance include the percentage of potential epilepsy surgery patients in whom the focus can be this precisely defined, including those with visible lesions or possible tandem foci; and whether, having identified optimal methods, combining them with cluster analysis may allow the application of these techniques to the IRSfMRI alone.

Having used infraslow EEG analysis to seed the epilepsy focus into the rsfMRI, we will continue mapping in fine detail the connections of the epilepsy network throughout the entire brain, and extend our techniques to additional patients. Through testing of multiple causal and graph theory parameters, we will identify those that best allow characterization of the epilepsy focus voxels as a discrete cluster. Then, using additional methods of cluster analysis, we hope that our techniques may allow the precise identification of the focus and network without seeding from the iEEG, but directly from the rsfMRI.